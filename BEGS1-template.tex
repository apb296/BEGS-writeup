\documentclass{beamer}
\usetheme{default}
\setbeamertemplate{navigation symbols}{}
%	
\usepackage{subfig}
\usepackage{amsmath, amsthm, amssymb}
\usepackage{float}
\usepackage{rotating}
\usepackage{graphicx}
\usepackage{longtable}
\usepackage{xcolor}
\usepackage{bm}
\usepackage{tikz}
\usetikzlibrary{shapes}
\tikzset{My Arrow Style/.style={single arrow, fill=red!50, anchor=base, align=center,text width=.5cm,rotate =270}}
\newcommand{\MyArrow}[2][]{\tikz[baseline] \node [My Arrow Style,#1] {#2};}
\tikzset{My 2Arrow Style/.style={single arrow, fill=red!50, anchor=base, align=center,text width=.5cm,rotate =90}}
\newcommand{\MyArrowUp}[2][]{\tikz[baseline] \node [My 2Arrow Style,#1] {#2};}
\newcommand{\bmat}{\begin{matrix}}
\newcommand{\emat}{\end{matrix}}
\newcommand{\EE}{\mathbb E}

\newtheorem{acknowledgement}[theorem]{Acknowledgement}
\newtheorem{algorithm}[theorem]{Algorithm}
\newtheorem{assumption}{Assumption}
\newtheorem{axiom}{Axiom}
\newtheorem{case}[theorem]{Case}
\newtheorem{claim}[theorem]{Claim}
\newtheorem{conclusion}[theorem]{Conclusion}
\newtheorem{condition}[theorem]{Condition}
\newtheorem{conjecture}{Conjecture}
\newtheorem{criterion}[theorem]{Criterion}
\newtheorem{proposition}{Proposition}
\newtheorem{summary}[theorem]{Summary}
\newtheorem{exercise}{Exercise}
\newtheorem{notation}{Notation}
\newtheorem{remark}{Remark}
%\graphicspath{{graphs//}}

\title { Optimal Fiscal
policies in some economies with incomplete asset markets}
\author{Anmol Bhandari, David Evans, Mikhail Golosov, Thomas J. Sargent}

\date{September 2013}
% \today will show current date.
% Alternatively, you can specify a date.
%
\begin{document}
%
\begin{frame}
\titlepage

\end{frame}

\begin{frame}
\frametitle{What do we do?}
We study optimal taxation under commitment with
\begin{itemize}
 \item \textbf{Representative agent}

 \item \textbf{Incomplete markets}

 \quad \color{red}$\rightarrow$ \color{black} assets with more general payoff structures

 \item \textbf{Linear tax schedules}

 \quad \color{red}$\rightarrow$ \color{black}Government levies a proportional tax on labor earnings 

 \item \textbf{Aggregate shocks}

 \quad \color{red}$\rightarrow$ \color{black} To productivities, government expenditure etc.

 \end{itemize}
\end{frame}


\begin{frame}
\frametitle{What are we after?}

\begin{enumerate}
\item Existing literature studies two extremes : economies with comple markets (LS) and economies with risk free bonds (AMSS)
\item We trace out intermediate cases where the asset traded provides partial spanning
 \item In particular study how the payoff structure affect long run properties of optimal government policies and equilibrium allocations?
\end{enumerate}

\end{frame}




\begin{frame}
 \frametitle{Environment}
 \begin{itemize}
 \item \textbf{Uncertainty}: Markov aggregate shocks $s_t$
  \item \textbf{Demography}: infinitely lived representative agent plus a benevolent planner
  \item \textbf{Technology}: Output  $\theta_{t} l_{t}$ is linear in labor supply
  \item \textbf{Preferences }(Households)
  \begin{equation*}
\mathbb{E}_{0}\sum_{t=0}^{\infty } \beta^t  U^{i}\left(
c_{i}(s^t),l_{i}(s^t)\right)  \label{utility lifetime}
\end{equation*}%
 \end{itemize}

\end{frame}

\begin{frame}
 \frametitle{Environment, II}
 \begin{itemize}
\item \textbf{Asset markets}: Private sector has complete markets; Government trades are retricted
  \item \textbf{Linear Taxes}: Agent $i$'s tax bill
\[- T_t + \tau_t \theta_{i,t}l_{i,t}\]

\item[]
  \item \textbf{Budget constraints}
  \begin{itemize}
   \item Agent $i$: $ c_{i,t}+b_{i,t}=\left( 1-\tau _{t}\right) \theta _{i,t}l_{i,t}+R_{t-1}b_{i,t-1}+T_{t}$
\item Government: $g_{t}+B_{t}+T_t=\tau _{t}\sum_{i=1}^{I}\pi _{i}\theta_{i,t}l_{i,t}+R_{t-1}B_{t-1}$
  \end{itemize}

\item[]
  \item \textbf{Market Clearing}
  \begin{itemize}
   \item Goods: $\sum_{i=1}^{I}\pi_{i}c_{i,t}+g_t =\sum_{i=1}^{I}\pi
_{i}\theta _{i,t} l_{i,t}$

   \item Assets: $\sum_{i=1}^{I}\pi _{i}b_{i,t}+B_{t}=0$
\end{itemize}
  \item[]

\item \textbf{Initial conditions}: Distribution of assets $\{b_{i,-1}\}_i$ and $B_{-1}$
\end{itemize}

\end{frame}


\begin{frame}
 \frametitle{Ramsey Problem}

\begin{definition}
\textbf{Allocation, price system, government policy}: Standard

\end{definition}

\begin{definition}
\textbf{Competitive equilibrium}: Given $\left( \left\{ b_{i,-1}\right\}
_{i},B_{-1}\right) $ and $\left\{ \tau _{t},T_{t}\right\} _{t=0}^{\infty }$
all allocations are chosen optimally, markets clear \footnote{Usually, we impose only  ``natural'' debt limits. }
\end{definition}

\begin{definition}
\textbf{Optimal competitive equilibrium}: A welfare-maximizing competitive
equilibrium for a given $\left( \left\{ b_{i,-1}\right\} _{i},B_{-1}\right) $
\end{definition}

 \end{frame}
 
 
 \begin{frame}
 \frametitle{Sequential problem}  
\[
	\max_{\{c_t,l_t,b_t\}} \EE_0\sum_{t=0}^\infty \beta^t U(c_t,l_t)
\]subject to
\begin{align*}
	\frac{b_{t-1}U_{c,t-1}}{\beta} = \frac{\EE_{t-1} p_t U_{c,t}}{p_t U_{c,t}}\EE_t\sum_{j=0}^\infty\beta^j\left( U_{c,t+j}c_{t+j}+U_{l,t+j}l_{t+j}\right)\text{  for $t\geq 1$ }\\
	b_{-1} = \frac1{U_{c,0}}\EE_0\sum_{t=0}^\infty \beta^t\left(U_{c,t}c_t+U_{l,t}l_t\right)\\
	c_t + g_t = \theta_t l_t
\end{align*}
Where $p_t$ is an iid random variable, representing an exogenous restriction on payoff structure of government assets.
  \end{frame}
  
 \begin{frame}
\frametitle{Optimal payoff structures}
\begin{itemize}
\item The planner's problem with state contingent debt can be represented as the problem above without the $t\geq1$ measurability restrictions.
\item  It can be thought of then as the planner choosing a payoff structure $p_t$, such that the measurability constraints for $t\geq 1$ are always satisfied.
\item  The optimal allocation then defines the optimal payoff structure chosen to be
\[
	p_t = \frac{\beta}{U_{c,t}b_{t-1}}\EE_t\sum_{j=0}^\infty\beta^j\left(U_{c,t+j}c_{t+j}+U_{l,t+j}l_{t+j}\right)
\]where
\[
	b_{t-1} = \frac{\beta}{U_{c,t-1}}\EE_{t-1}\sum_{j=0}^\infty\beta^j\left(U_{c,t+j}c_{t+j}+U_{l,t+j}l_{t+j}\right)
\]under the normalization $\frac{\EE_{t-1} p_t U_{c,t}}{U_{c,t-1}} = 1$
\end{itemize}
\end{frame} 
 

\begin{frame}
	\frametitle{Quasilinear FOC}
	The familiar first order conditions for the planner's problem where preferences are given by $U(c,l) = c - \frac{l^{1+\gamma}}{1+\gamma}$ are
	\begin{align*}
		1 = \mu +\xi_t\\
		-l_t^\gamma+\mu(1+\gamma)l_t^\gamma+\theta_t\xi_t = 0
	\end{align*}Where $\mu$ is the Lagrange multiplier on the implimentability constraint and $\beta^t\xi_t$ is the multiplier of the resource constraint.   These equations give the tax rate $\tau$ as a function of $\mu$
	\[
		\tau(\mu) = \frac{\gamma\mu}{(1+\gamma)\mu-1}
	\] and the period government surplus 
	\[
		S_t(\tau) = \theta_t^\frac\gamma{1+\gamma}(1+\tau)^\frac1\gamma\tau-g_t
	\]
\end{frame}

\begin{frame}
	\frametitle{Quasilinear Asset Structure}
	Under the i.i.d. assumption, the optimal payoff structure chosen by the planner is given by the equation
	\[
		p = (1-\beta)\frac{St(\tau)}{\EE S(\tau)} + \beta
	\]where $\tau$ is the optimal tax rate given government debt $b$. 
	\begin{lemma}  Let $p_{\theta,g}(b)$ be the payoff chosen by the planner in aggregate state $(\theta,g)$ when government debt is given by $b$.  Then $p_{\theta,g}(b)$ will, generically, be a one-to-one function of $b$.  If $g\EE\theta^\frac{\gamma}{1+\gamma}-\theta^\frac{\gamma}{1+\gamma}\EE g > (<) 0$ then $p_{\theta,g}(b)$ will be an increasing (decreasing) function of $b$. 
	\end{lemma}
\end{frame}

  \begin{frame}
   \frametitle{Graphs }
    a graph that explains the main parts of the theorem
   
   - payoff structures and taxes as a function of initial debt
  \end{frame}

  
  
\begin{frame}
 	\frametitle{Incomplete Markets Solution}
	With quasilinear preferences the measurability constraints for the incomplete markets problem can be written as the period by period constraint
	\[
		\frac{p_t b_{t-1}}\beta = c_t - l_t^{1+\gamma} + b_t 
	\]under the normalization of $\EE_{t-1} p_t = 1$  The first order conditions associated with this problem are
	\begin{align*}
		1 = \mu_t + \xi_t\\
		-l_t^\gamma+\mu_t(1+\gamma)l_t^\gamma+\theta_t\xi_t = 0\\
		\mu_t = \EE_t p_{t+1}\mu_{t+1}
	\end{align*}  Where $\beta^t\mu_t$ is the multiplier on the period by period budget constraint and $\beta^t\xi_t$ is the multiplier on the resource constraint.
\end{frame}


\begin{frame}
 \frametitle{Steady states}
\begin{itemize}
	\item  We will restrict ourselves to an iid economy.
	\item  If there existed a steady state then $\mu_t = \mu$ for all periods and $b_t = b$.
	\item  The first order conditions for the incomplete markets economy then simplify to
	\begin{align*}
		 1 = \mu+ \xi\\
		-l_t^\gamma + \mu_t(1+\gamma)l_t^\gamma +\theta_t\xi = 0
	\end{align*}
	\item  Note these are the exact same first order conditions governing the complete markets solution.
	\item  Thus an incomplete markets steady state allocation replicates a complete markets solution. 
	\item  Finding a steady state amounts to finding a complete markets optimal allocation who's optimal payoff structure matches $p_t$
\end{itemize}
\end{frame}

\begin{frame}
 \frametitle{Existence }
 
 - Impose SS on FOC of AMSS = FOC of LS

 - apply the LS theorem

 - mention that SS is just the selection of a particular initial condition for a LS economy
 
\end{frame}

\begin{frame}
 \frametitle{Characterization}
- Rexamine payoffs and sign of portfolio
 \end{frame}

 \begin{frame}
  \frametitle{Intuition}
  - future implementability constraints are relaxed
 \end{frame}

 \begin{frame}
  \frametitle{Risk adjusted martingales}
 \end{frame}
 
 \begin{frame}
  \frametitle{Convergence}
  - theorem (1) quasilinear
 
 - theorem (2) extensions to risk aversion
 
 - remarks on more general shock structures
 \end{frame}

 
 \begin{frame}
  \frametitle{how does it work}
  undoing the risk adjustment
 \end{frame}

 
 
 \begin{frame}
  \frametitle{Role of transfers}
 \end{frame}

 \begin{frame}
 	\frametitle{Incomplet Markets Solution}
	By choosing a new state variable $x_{t-1} = b_{t-1} U_{c,t-1}$ the measurability constraints can be written as a period by period constraint
	\[
		\frac{x_{t-1} p_t U_{c,t}}{\beta \EE_{t-1} p_t U_{c,t}}  = U_{c,t}c_t+U_{l,t} l_t + x_t
	\]  Which allows us to write the problem recursively as follows
	V
\end{frame}
  
 \begin{frame}
  \frametitle{numerical section}
   for realistic calibrations
  - speed of convergence
  - nature of convergence when there are no steady states
 \end{frame}
 
 
 \begin{frame}
  \frametitle{some simulations}
  \end{frame}
 
 \begin{frame}
  \frametitle{conclusion}
 \end{frame}

 \end{document}
 