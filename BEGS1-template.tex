\documentclass{beamer}
\usetheme{default}
\setbeamertemplate{navigation symbols}{}
%	
\usepackage{subfig}
\usepackage{amsmath, amsthm, amssymb}
\usepackage{float}
\usepackage{rotating}
\usepackage{graphicx}
\usepackage{longtable}
\usepackage{xcolor}
\usepackage{bm}
\usepackage{tikz}
\usetikzlibrary{shapes}
\tikzset{My Arrow Style/.style={single arrow, fill=red!50, anchor=base, align=center,text width=.5cm,rotate =270}}
\newcommand{\MyArrow}[2][]{\tikz[baseline] \node [My Arrow Style,#1] {#2};}
\tikzset{My 2Arrow Style/.style={single arrow, fill=red!50, anchor=base, align=center,text width=.5cm,rotate =90}}
\newcommand{\MyArrowUp}[2][]{\tikz[baseline] \node [My 2Arrow Style,#1] {#2};}
\newcommand{\bmat}{\begin{matrix}}
\newcommand{\emat}{\end{matrix}}

\newtheorem{acknowledgement}[theorem]{Acknowledgement}
\newtheorem{algorithm}[theorem]{Algorithm}
\newtheorem{assumption}{Assumption}
\newtheorem{axiom}{Axiom}
\newtheorem{case}[theorem]{Case}
\newtheorem{claim}[theorem]{Claim}
\newtheorem{conclusion}[theorem]{Conclusion}
\newtheorem{condition}[theorem]{Condition}
\newtheorem{conjecture}{Conjecture}
\newtheorem{criterion}[theorem]{Criterion}
\newtheorem{proposition}{Proposition}
\newtheorem{summary}[theorem]{Summary}
\newtheorem{exercise}{Exercise}
\newtheorem{notation}{Notation}
\newtheorem{remark}{Remark}
%\graphicspath{{graphs//}}

\title { Optimal Fiscal
policies in some economies with incomplete asset markets}
\author{Anmol Bhandari, David Evans, Mikhail Golosov, Thomas J. Sargent}

\date{September 2013}
% \today will show current date.
% Alternatively, you can specify a date.
%
\begin{document}
%
\begin{frame}
\titlepage

\end{frame}

\begin{frame}
\frametitle{What do we do?}
We study optimal taxation under commitment with
\begin{itemize}
 \item \textbf{Representative agent}

 \item \textbf{Incomplete markets}

 \quad \color{red}$\rightarrow$ \color{black} assets with more general payoff structures

 \item \textbf{Linear tax schedules}

 \quad \color{red}$\rightarrow$ \color{black}Government levies a proportional tax on labor earnings 

 \item \textbf{Aggregate shocks}

 \quad \color{red}$\rightarrow$ \color{black} To productivities, government expenditure etc.

 \end{itemize}
\end{frame}


\begin{frame}
\frametitle{What are we after?}

\begin{enumerate}
\item Existing literature studies two extremes : economies with comple markets (LS) and economies with risk free bonds (AMSS)
\item We trace out intermediate cases where the asset traded provides partial spanning
 \item In particular study how the payoff structure affect long run properties of optimal government policies and equilibrium allocations?
\end{enumerate}

\end{frame}




\begin{frame}
 \frametitle{Environment}
 \begin{itemize}
 \item \textbf{Uncertainty}: Markov aggregate shocks $s_t$
  \item \textbf{Demography}: infinitely lived representative agent plus a benevolent planner
  \item \textbf{Technology}: Output  $\theta_{t} l_{t}$ is linear in labor supply
  \item \textbf{Preferences }(Households)
  \begin{equation*}
\mathbb{E}_{0}\sum_{t=0}^{\infty } \beta^t  U^{i}\left(
c_{i}(s^t),l_{i}(s^t)\right)  \label{utility lifetime}
\end{equation*}%
 \end{itemize}

\end{frame}

\begin{frame}
 \frametitle{Environment, II}
 \begin{itemize}
\item \textbf{Asset markets}: Private sector has complete markets; Government trades are retricted
  \item \textbf{Linear Taxes}: Agent $i$'s tax bill
\[- T_t + \tau_t \theta_{i,t}l_{i,t}\]

\item[]
  \item \textbf{Budget constraints}
  \begin{itemize}
   \item Agent $i$: $ c_{i,t}+b_{i,t}=\left( 1-\tau _{t}\right) \theta _{i,t}l_{i,t}+R_{t-1}b_{i,t-1}+T_{t}$
\item Government: $g_{t}+B_{t}+T_t=\tau _{t}\sum_{i=1}^{I}\pi _{i}\theta_{i,t}l_{i,t}+R_{t-1}B_{t-1}$
  \end{itemize}

\item[]
  \item \textbf{Market Clearing}
  \begin{itemize}
   \item Goods: $\sum_{i=1}^{I}\pi_{i}c_{i,t}+g_t =\sum_{i=1}^{I}\pi
_{i}\theta _{i,t} l_{i,t}$

   \item Assets: $\sum_{i=1}^{I}\pi _{i}b_{i,t}+B_{t}=0$
\end{itemize}
  \item[]

\item \textbf{Initial conditions}: Distribution of assets $\{b_{i,-1}\}_i$ and $B_{-1}$
\end{itemize}

\end{frame}


\begin{frame}
 \frametitle{Ramsey Problem}

\begin{definition}
\textbf{Allocation, price system, government policy}: Standard

\end{definition}

\begin{definition}
\textbf{Competitive equilibrium}: Given $\left( \left\{ b_{i,-1}\right\}
_{i},B_{-1}\right) $ and $\left\{ \tau _{t},T_{t}\right\} _{t=0}^{\infty }$
all allocations are chosen optimally, markets clear \footnote{Usually, we impose only  ``natural'' debt limits. }
\end{definition}

\begin{definition}
\textbf{Optimal competitive equilibrium}: A welfare-maximizing competitive
equilibrium for a given $\left( \left\{ b_{i,-1}\right\} _{i},B_{-1}\right) $
\end{definition}

 \end{frame}
 
 
 \begin{frame}
 \frametitle{Sequential problem}  
  \end{frame}
  
  
 
 \begin{frame}
  \frametitle{Optimal payoff structures}
Study comparative statics in LS
1] quasi linear LS
 
\begin{enumerate}
 \item FOC
 \item Theorem for quasi linear
 \item Extentions for CES
\end{enumerate}

  \end{frame}
  
  \begin{frame}
   \frametitle{Graphs }
    a graph that explains the main parts of the theorem
   
   - payoff structures and taxes as a function of initial debt
  \end{frame}

  
  
\begin{frame}
 \frametitle{Recursive AMSS}
\end{frame}


\begin{frame}
 \frametitle{Steady states}
\end{frame}

\begin{frame}
 \frametitle{Existence }
 
 - Impose SS on FOC of AMSS = FOC of LS

 - apply the LS theorem

 - mention that SS is just the selection of a particular initial condition for a LS economy
 
\end{frame}

\begin{frame}
 \frametitle{Characterization}
- Rexamine payoffs and sign of portfolio
 \end{frame}

 \begin{frame}
  \frametitle{Intuition}
  - future implementability constraints are relaxed
 \end{frame}

 \begin{frame}
  \frametitle{Risk adjusted martingales}
 \end{frame}
 
 \begin{frame}
  \frametitle{Convergence}
  - theorem (1) quasilinear
 
 - theorem (2) extensions to risk aversion
 
 - remarks on more general shock structures
 \end{frame}

 
 \begin{frame}
  \frametitle{how does it work}
  undoing the risk adjustment
 \end{frame}

 
 
 \begin{frame}
  \frametitle{Role of transfers}
 \end{frame}

 
  
 \begin{frame}
  \frametitle{numerical section}
   for realistic calibrations
  - speed of convergence
  - nature of convergence when there are no steady states
 \end{frame}
 
 
 \begin{frame}
  \frametitle{some simulations}
  \end{frame}
 
 \begin{frame}
  \frametitle{conclusion}
 \end{frame}

 \end{document}
 